
\clearpage

\subsection{}
First we should plot the neutral voltage. That makes things easier.

\begin{center}
   \begin{tikzpicture}[]
\begin{axis}[standard,
             xlabel={$\omega t$},
             ylabel={$V_{n}$},
             xtick={60,120,180,240,300,360,420}, 
             xticklabels={$\frac{\pi}{3}$,
                          $\frac{2\pi}{3}$,
                          $\pi$,
                          $\frac{4\pi}{3}$,
                          $\frac{5\pi}{3}$,
                          $2\pi$,
                          $\frac{7\pi}{3}$
                          },
             ytick={-1,1},
             yticklabels={$-\frac{V_{dc}}{3}$,
                          $\frac{V_{dc}}{3}$
                          },
             domain=0:430,
             width=0.8\textwidth,
             height=5cm,
             ymax=1.2,
             ymin=-1.2,
             ] 
    \addplot[blue, samples=430] 
    coordinates{
    (0,1)
    (60,1)
    (60,-1)
    (120,-1)
    (120,1)
    (180,1)
    (180,-1)
    (240,-1)
    (240,1)
    (300,1)
    (300,-1)
    (360,-1)
    (360,1)
    (420,1)
    (420,-1)
    (430,-1)
    }
    ; 
    % \legend{$T_{#1}$}
\end{axis}
\end{tikzpicture}
\end{center}


\pgfmathdeclarefunction{lineNV}{1}{%
  \pgfmathparse{%
    (2*and(1,mod(#1+360,360) < 180) - 1 )
   * (1 +  and(mod(#1+360,180) < 120,mod(#1+360,180) > 60)) * 0.5   
  }%
}

\newcommand\lineNVPlot[2]{

\begin{tikzpicture}[]
\begin{axis}[standard,
             xlabel={$\omega t$},
             ylabel={$V_{#2}$},
             xtick={60,120,180,240,300,360,420}, 
             xticklabels={$\frac{\pi}{3}$,
                          $\frac{2\pi}{3}$,
                          $\pi$,
                          $\frac{4\pi}{3}$,
                          $\frac{5\pi}{3}$,
                          $2\pi$,
                          $\frac{7\pi}{3}$
                          },
             ytick={-1,-0.5,0.5,1},
             yticklabels={$-\frac{2V_{dc}}{3}$,
                          $-\frac{V_{dc}}{3}$,
                          $\frac{V_{dc}}{3}$,
                          $\frac{2V_{dc}}{3}$
                          },
             domain=0:430,
             width=0.8\textwidth,
             height=6cm,
             ymax=1.2,
             ymin=-1.2,
             ] 
    \addplot[blue, samples=430] {lineNV(x-#1)}; 
    % \legend{$T_{#1}$}
\end{axis}
\end{tikzpicture}
}

\begin{centering}
   \foreach \wt/\l in {0/{A},
                       120/{B},
                       240/{C}
                       }{
   
      \lineNVPlot{\wt}{\l}
      
   }
\end{centering}