\subsection{}

No.

\todo[inline]{What's a good explanation?}

\subsubsection*{Mathematical Proof}

From KCL,

\begin{equation}
\label{eq:1b diff}
V_s \sin(\omega t) = R i(t) + L\frac{di(t)}{dt}
\end{equation}

The current must satisfy that equation.

We already know a solutions is

\begin{align*}
i(t) & = i_{ss} + i_t \\
     & = \frac{V_s}{z} \sin(\omega t - \Phi) + \sin{\Phi} e^{-\frac{Rt}{L}}
\end{align*}

Let's try to shift the curve up so that it starts at a non-zero value.

\begin{equation}
\label{eq:1b guess}
i(t) = \frac{V_s}{z} \sin(\omega t - \Phi) + \frac{V_s}{z} \sin{\Phi} e^{-\frac{Rt}{L}} + C
\end{equation}

If we sub \cref{eq:1b guess} into \cref{eq:1b diff}, we see that it doesn't hold.


\subsubsection*{Intuitive Explanation}

We know that current between $\alpha$ and $\beta$ is given by a sinusiod (steady state solution) plus an exponential decay. 

$$
i(t) = \frac{V_s}{z} \sin(\omega t - \Phi) + \frac{V_s}{z} \sin{\Phi} e^{-\frac{Rt}{L}} 
$$

For non-zero $R$, the amplitude of the exponential decay is less than the amplitude of the sinusoid. Therefore the curve must cross the origin twice each period.


\subsubsection*{Smart Arse Answer}

The nomograph for the extinction angle (the $\beta$, $\alpha$ graph) only reaches $\beta{}=360^\circ$ for $\Phi=90^\circ$.