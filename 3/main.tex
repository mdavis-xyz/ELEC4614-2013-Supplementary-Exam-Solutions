\section{Non-Isolated Step Down Converter}

We could use either a buck or a boost converter.

I'm choosing a buck. The reasons is that oost converters have output stage current. This means that the RMS capacitor current is far higher, so there is higher ripple.
 
\subsection*{Inductor Value}

\begin{align*}
I_L & = I_o \\
\Delta I_L & = \frac{V_o (1-D)T_s}{L} \\
\intertext{For DCM/CCM boundary}
I_L & = \frac{\Delta I_L}{2} \\
I_o & = \frac{V_o (1-D)T_s}{2L} \\
Lf_s & = \frac{R(1-D)}{2} \\
\intertext{The worst case scenario is that $I_o$ is at it's minimum.}
L & = \frac{V_o (1-D)}{2f_sI_{L_{min}}} \\
\intertext{This must be true of all $D$, so choose minimum $D$ for worst case scenario, so choose maximum $V_d$.}
L & = \frac{V_o (1-D_{min})}{2f_sI_{L_{min}}} \\
  & = \frac{V_o (1-\frac{V_o}{V_{d_{max}}})}{2f_sI_{L_{min}}} \\
  & = \frac{5 \times (1-\frac{5}{12})}{2\times 20k \times 0.1} \\
  & = \unit[729.17]{\mu{} H}
\end{align*}

\subsection*{Capacitance Value}

\begin{align*}
Q & = \frac{1}{2} \times \frac{T_s}{2} \times \frac{\Delta I_L}{2} \\
  & = \frac{1}{8f_s} \times \frac{V_d (1-D) T_s}{L} \\
  & = \frac{V_o (1-D)}{8f_s^2 L} \\
C & = \frac{Q}{\Delta V} \\
  & = \frac{\frac{V_o (1-D)}{8f_s^2 L}}{\Delta V} \\
\intertext{We care about the worst case, which is the largest $C$, so smallest $D$, so biggest $V_d$,}
  & = \frac{5 \times (1-\frac{5}{12})}{8 \times 20k^2 \times 729.17\mu 20m} \\
  & = \unit[62.5]{\mu F}
\end{align*}

\subsection*{Peak Current Ratings}

\begin{align*}
I_{max} & = I_L + \frac{\Delta I_L}{2} \\
\intertext{We want the worst case scenario, so maximise $I_{max}$, so minimize $D$, which means maximize $V_d$, and also maximize $I_o$,}
        & = I_{o_{max}} + \frac{V_o (1-D)T_s}{2L} \\
        & = I_{o_{max}} + \frac{V_o (1-\frac{V_o}{V_{d_{max}}})}{2Lf_s} \\ 
        & = 1 + \frac{5 \times \left( 1 - \frac{5}{12}\right)}{2 \times 20k} \\
        & = \unit[1.1]{A}
\end{align*}

That's for the diode and the switch.
Maybe make it \unit[1.2]{A}, just so there's a bit of head room.

\subsection*{Duty Range}

For $V_d=\unit[8]{V}$,

\begin{align*}
D & = \frac{V_o}{V_d} \\
  & = \frac{5}{8} \\
  & = 0.625
\end{align*}


For $V_d=\unit[12]{V}$,

\begin{align*}
D & = \frac{V_o}{V_d} \\
  & = \frac{5}{12} \\
  & \approx 0.417
\end{align*}

So $D$ ranges from 0.417 to 0.625

\subsection*{Demonstrate CCM is guarenteed}

From earlier, we showed that to obtain CCM we need

$$
Lf_s \geq \frac{R(1-D)}{2} = \frac{V_o \left( 1 - \frac{V_o}{V_d}\right)}{2 I_o}
$$

The left hand side is

\begin{align*}
LHS  & = Lf_s \\
     & = 729.17 \mu \times 20k \\
     & = 14.58
\end{align*}

For $V_d=\unit[8]{V}$ and $I=\unit[1]{A}$:

\begin{align*} 
RHS   & = \frac{V_o \left( 1 - \frac{V_o}{V_d}\right)}{2 I_o} \\
      & = \frac{5 \times \left( 1 - \frac{5}{8}\right)}{2 \times 1} \\
      & = 0.9375
      & \leq LHS
\end{align*}

So CCM is guaranteed for $V_d=\unit[8]{V}$ and $I=\unit[1]{A}$.


For $V_d=\unit[8]{V}$ and $I=\unit[0.1]{A}$:

\begin{align*} 
RHS   & = \frac{V_o \left( 1 - \frac{V_o}{V_d}\right)}{2 I_o} \\
      & = \frac{5 \times \left( 1 - \frac{5}{8}\right)}{2 \times 0.1} \\
      & = 9.375
      & \leq LHS
\end{align*}

So CCM is guaranteed for $V_d=\unit[8]{V}$ and $I=\unit[0.1]{A}$.


For $V_d=\unit[12]{V}$ and $I=\unit[1]{A}$:

\begin{align*} 
RHS   & = \frac{V_o \left( 1 - \frac{V_o}{V_d}\right)}{2 I_o} \\
      & = \frac{5 \times \left( 1 - \frac{5}{12}\right)}{2 \times 1} \\
      & = 1.4583
      & \leq LHS
\end{align*}

So CCM is guaranteed for $V_d=\unit[12]{V}$ and $I=\unit[1]{A}$.


For $V_d=\unit[12]{V}$ and $I=\unit[0.1]{A}$:

\begin{align*} 
RHS   & = \frac{V_o \left( 1 - \frac{V_o}{V_d}\right)}{2 I_o} \\
      & = \frac{5 \times \left( 1 - \frac{5}{12}\right)}{2 \times 0.1} \\
      & = 14.58
      & \leq LHS
\end{align*}

So CCM is guaranteed for $V_d=\unit[12]{V}$ and $I=\unit[0.1]{A}$:

Therefore CCM is guarenteed for all specified conditions.

\subsection*{Assumptions}

\begin{itemize}
    \item The diode is ideal
    \begin{itemize}
        \item No resistance
        \item No forward voltage
        \item No reverse recovery
    \end{itemize}
    \item The inductor is ideal
    \begin{itemize}
        \item No resistance
    \end{itemize}
    \item The capacitor is ideal
    \begin{itemize}
        \item No resistance
    \end{itemize}
\end{itemize}

\todo[inline]{What have I forgotten? I'm sure there's more.}